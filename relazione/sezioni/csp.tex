% !TEX encoding = UTF-8
% !TEX TS-program = pdflatex
% !TEX root = ../relazione.tex
% !TEX spellcheck = it-IT
\clearpage
\section{N-Regine come CSP}\label{app:csp}

Il problema delle $N$-Regine può essere anche formulato come problema di soddisfacimento di vincoli, andando a codificare l'assenza di minacce con dei vincoli definiti sui possibili assegnamenti delle variabili. Una possibile formulazione è:
\\ \\
\textit{Variabili:}
$$
x_i \in \{0 \ldots N - 1\} \qquad i = 0 \ldots N-1
$$
\\ \\
\textit{Vincoli:} $\qquad \qquad  \qquad \qquad  \qquad \forall i, j \in \{0 \ldots N - 1\}, \quad i \neq j$
\begin{align*}
x_i &\neq  x_j  \\
x_i - x_j &\neq j - i \\
x_i - x_j &\neq i - j  \\
\end{align*}

Con il problema così codificato è possibile utilizzare un risolutore di CSP che cerca di produrre un assegnamento di variabili in modo che tutti i vincoli siano soddisfatti.

\`{E} stata fatta una prova utilizzando or-tools, un risolutore CSP manutenuto da Google, per verificare l'efficacia di questo approccio e per paragonarlo alla ricerca locale.

\`{E} risultato che per risolvere $26$-Regine sotto forma di CPS e utilizzando il risolutore di default, sono necessari $846 ms$,
Andando a modificare il criterio con il quale il risolutore assegna un valore alle variabili in modo che vengano assegnate prima le variabili con il dominio più piccolo e assegnate al valore più basso possibile, si riesce ad ottenere una soluzione in $5 ms$ e in circa $5$ secondi si ottiene una soluzione per $300$-Regine.

Questa differenza di prestazioni deriva dal fatto che il risolutore CSP riesce ad utilizzare i vincoli per effettuare una riduzione dei domini delle variabili, riducendo così lo spazio di ricerca. 
Inoltre è possibile ottimizzare la strategia di ricerca in modo che vengano assegnate per prime le variabili per cui si ha meno scelta, ovvero con un dominio più piccolo. 
Così facendo il risolutore riesce a riconoscere prima se il ramo dell'albero che sta valutando porta ad una soluzione o meno.

Con la ricerca locale, l'unica informazione che si ha a disposizione è la funzione euristica, la quale fornisce meno informazioni rispetto ai vincoli di un problema con vincoli.
Quindi può succedere che la ricerca locale esplori un'insieme di stati che il risolutore CSP riesce a scartare perché rileva subito che non sono soluzioni.
Inoltre, la ricerca locale è influenzata dallo stato iniziale: se vicino allo stato iniziale c'è uno stato a cui corrisponde un minimo locale della funzione euristica, l'algoritmo andrà probabilmente a bloccarsi in quello stato, mentre con un risolutore CSP non c'è questo problema perché lo stato iniziale è sempre lo stato in cui nessuna variabile è assegnata.

Infine, $N$-Regine rientra nella categoria dei problemi combinatori, problemi che sono facilmente risolvibili con la programmazione a vincoli.

\subsection{Enumerazione delle soluzioni di N-Regine}

Un'altra caratteristica dei CSP è che una volta definito il modello, questo può essere utilizzato sia per trovare una soluzione ottima che per enumerare tutte le possibili soluzioni.

Si è quindi utilizzato lo stesso modello per contare quante sono le soluzioni possibili delle istanze di $N$-Regine affrontate nel progetto. 
Purtroppo l'enumerazione è molto più complessa in quanto il risolutore deve attraversare tutto l'albero di ricerca e non può fermarsi alla prima soluzione trovata, pertanto l'enumerazione delle soluzioni è stata effettuata fino a $15$-Regine.

Il risultato ottenuto è riportato nella tabella \ref{table:sol}, dalla quale è possibile notare la crescita esponenziale del numero di soluzioni e che $6$-Regine ha meno soluzioni di $5$-Regine.

\begin{table}[]
\centering
\resizebox{\textwidth}{!}{%
\begin{tabular}{|c|c|c|c|c|c|c|c|c|c|c|c|c|c|c|c|}
\hline
\textbf{N}   & 1 & 2 & 3 & 4 & 5  & 6 & 7  & 8  & 9   & 10  & 11   & 12    & 13    & 14     & 15      \\ \hline
\textbf{\#Sol} & 1 & 0 & 0 & 2 & 10 & 4 & 40 & 92 & 352 & 724 & 2680 & 14200 & 73712 & 365596 & 2279184 \\ \hline
\end{tabular}
}
\caption{Numero di soluzioni per $N$-Regine}
\label{my-label}
\end{table}


